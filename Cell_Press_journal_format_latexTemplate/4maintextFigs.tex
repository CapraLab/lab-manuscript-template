% Create a "code name" for each figure. In the maintext, when you want that figure to appear in the manuscript call that function (e.g. \figCodeName).

%%%% Create and format Figure with subpanels: %%%%
% \generateFigSubpanels[optionalRotationDegrees]{figCodeName}{figureNamePath}{size:propOfTextwidth}
%     {main caption}
%     {{
%         {subcaption1},
%         {subcaption2},
%         {subcaption3}
%     }}

% Then in the reference your figures with \cref{fig:figCodeName} and Reference subpanels with \cref{fig:figCodeName:A}


%%%% Function to create and format figures without subpanels %%%%

% \generateFig[optionalRotationDegrees]{figCodeName}{figureNamePath}{size:propOfTextwidth}
%     {Bold text main caption}
%     {Smaller normal text caption}

% Then in the main text reference your figures with \cref{fig:figCodeName}

% You can also use \generateSidwaysFigSubpanels or \generateSidewaysFig to rotate the caption with the figure.

%%%% Example figure with codename: figureSuggestions %%%%

\generateFigSubpanels{figureSuggestions}{figureSuggestions.pdf}{1}
    {Figure title which should not end in a period} % Main caption
    {{
        {Succinct caption for Panel A. Note that figure captions will not split across pages.}, % Subcaption 1
        {Caption for panel B. Add additional subpanel captions as needed depending on the figure.} % Subcaption 2
    }}