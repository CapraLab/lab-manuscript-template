\section{DISCUSSION}
\discussionTips % Delete when no longer needed
Per \textit{Cell Genomics}: ``The discussion should explain the significance of the results and place them into a broader context. It is often helpful to the reader to indicate the directions in which the work might be built on going forward. It should not be redundant with the results. The discussion may contain subheadings and can be combined with the results section.''

\textbf{First paragraph} Brief summary of the paper that sets up the discussion of the major contributions. You can start by restating the major gap that you addressed (\textit{Despite X, we previously had a limited understanding of Y. Here, we show/demonstrate Z. Together, these findings highlight/provide\dots}) This can parallel the last paragraph of your intro (where you outline the big picture questions and their context for the field) or parallel the results, but this is not required.

\textbf{Middle Paragraph(s)} (1-4 paragraphs) Major contribution/finding/conclusion and its context. What does your result/contribution mean? Who does it speak to and why do they care? What are the implications? Were your hypotheses correct, surprising, contradictory? If the results are not conclusive, why not? How would you perform further work to get to a more conclusive result?

\textbf{Penultimate paragraph(s)} (1-2 paragraphs) Limitations. Identify the broad categories/ themes of limitations and don’t get too particular about the details of each limitation. (i.e., need more samples/ data, model misspecification or over-simplification). Keep it succinct, don’t apologize, but show that you have thought about caveats. Should segue into the future directions. \textit{Although our approach provides many novel benefits, it also has limitations that we hope future work will address.} 

\textbf{Last paragraph} Future directions and final big picture conclusion. This could be broken out into a separate conclusions section. \textit{As X happens, we anticipate that Y will facilitate understanding of Z which allows for W}. Why is this exciting for your audience? Where are the next steps? Note that this is could possibly result in scooped, but even if you are thinking of working on a next step, consider putting it out there, because science for all!! \textit{We encourage future studies to address X}. Be positive about specific future opportunities inspired by your work. It leaves an impression).