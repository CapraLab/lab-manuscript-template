\section{Introduction}
\introTips % Delete the tips

\noindent\textbf{First paragraph(s)} What is the problem? Why do we care?  First sentence can be a general truth or general limitation that your audience accepts. E.g., \textit{While differences between archaic and modern humans are well described, there is a poor understanding of the genetic mechanisms underlying such phenotypic differences.} or  \textit{Most aspects of archaic hominin biology cannot be directly studied due to their lack of preservation in fossils.} Don't waste the first sentence with a definition.

\textbf{Middle paragraph(s)} What have others found about this specific problem? What are the current hypotheses and support for each? and/or what data have been previously analyzed? What are the gaps? E.g., \textit{One such mechanism may be alternative splicing, but we cannot obtain such data from extinct taxa.} This part of the Intro may require a few paragraphs, but you do not need to fully review the history of work on the question. Focus on the current state of the field.

\textbf{Last paragraph} Overview of the major questions addressed here and why they are important. \textit{Here, we investigated X using dataset Y and dataset Z. We tested the A hypothesis. We also considered this other obscure hypothesis. We predicted x, y, and z.} E.g., \textit{Here, we leverage a new deep neural network that can predict splice altering variants from sequence alone to examine such variants in archaic hominins.}
