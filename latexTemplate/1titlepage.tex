%%% TITLE, AUTHOR, AFFILIATIONS, ABSTRACT %%%

%%% ------------ TITLE ------------ %%%

\title{The Capra Lab manuscript template\titleTips}

%%% ------------ AUTHORS ------------ %%%

\author[1]{Your~Name~Here}
\author[2,3,*]{John~A.~Capra}
\affil[1]{Place, City, XX}
\affil[2]{Bakar Computational Health Sciences Institute, University of California, San Francisco, CA}
\affil[3]{Department of Epidemiology and Biostatistics, University of California, San Francisco, CA}

\affil[ ]{ } % for some blank space
\affil[*]{\textit{Correspondence to }\underline{tony@capralab.org}}
\renewcommand\Affilfont{\footnotesize} % size of affilitations
\setcounter{Maxaffil}{0} % no max for num affiliations
\date{} % If you want a date fill this in (e.g. \date{February 2022})

%%% ------------ ABSTRACT ------------ %%%

\newcommand{\makeAbstract}{
\begin{abstract}
\abstractTips

\paragraph{Background}
\noindent (No more than three sentences.) Motivate your work by stating the problem you addressed or the hypothesis you tested. The last sentence should clearly state the gap in knowledge.

\paragraph{Methods}
(One or two optional sentences; this can be combined with results if it is not a methods focused paper. \textit{To address this gap, we applied X to Y.} or \textit{we synthesized V and W to reveal Z.}

\paragraph{Results}
(Most of the rest of the abstract) \textit{Here, we show\dots} What did you do and what did you find? This should summarize the main analyses and results of your work. Keep it high level; it is ok if some results are not mentioned.

\paragraph{Conclusions}
(No more than two sentences.) What did you learn from your studies? How does this change the field? What do your results say about the problem/hypothesis identified in the Background section?
\end{abstract}

}

