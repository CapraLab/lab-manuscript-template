%%% ------------ OVERVIEW OF THIS TEMPLATE ------------ %%%
% This is a Capra Lab Manuscript template! It was developed by the Capra
% Lab and converted to a LaTex document by Evonne McArthur 2/2022 and last
% edited 3/2022. 

% To use this template, you should probably read the whole compiled document
% because it contains lots of great tips. Once you are ready to go, you 
% should delete the tips in each section (\abstractTips \introTips \resultsTips 
% \captionTips \methodsTips \methodsDataTips, \methodsAnalysisTips \discussionTips and \generalTips). Then you should
% delete the REMOVEME_tips.tex file and the call to that file in this document
% (%%%% Delete this file to remove all the tips %%%%
% Make sure you delete all the \[section]Tips in the main text files!!

\newenvironment{tips} % To make the blue/gray-colored, small text environment to format the tips
    {
        \footnotesize\color{CadetBlue}
        \begin{adjustwidth}{1cm}{1cm}
        \begin{singlespace}
        \vspace{.5em}
        \nolinenumbers
    }
    {
        \end{singlespace}
        \end{adjustwidth}
        \vspace{.5em}
        %\ignorespacesafterend
    }
    

    
\newcommand{\titleTips}
    {
    \begin{tips}
    \centering
        State a result with the title if possible. Aim for fewer than 10 words.
    \end{tips}
    }
    
\newcommand{\abstractTips}
    {
    \begin{tips}
        \noindent $<$ 250 words. Shorter is better. Check journal limits, structured or unstructured, and so on. \medskip
        
        \noindent Make a compelling elevator pitch for the paper that includes a brief intro, methods, results, conclusions. It can be split into sections or not. Always start with sections to provide a framework, but remove if needed. Revisit the abstract as you write the paper (even if you have written abstracts for conferences previously), because you will find better ways to summarize the paper as it evolves! 
        
        First sentence should make a broad and general statement that sets up to the importance of the topic. Avoid definitions in the first sentence. For example, don’t say \textit{TADs are 3D structures.} It is boring and does not convey importance or a gap. The first sentence can be the same as the first sentence of the Intro but often is more specific/less broad. Overall, it should establish the gap in knowledge as briefly as possible and avoid too much background. Put your main question/gap early (2\textsuperscript{nd} or 3\textsuperscript{rd} sentence). End the abstract by making the significance and implications for the field (and maybe next steps) clear. Be as broad as possible.  No citations and minimize explicit references to other work in the abstract (meta-discourse like \textit{Previous work that investigated X showed Y}); instead, just talk about Y and the gap that you will fill.
    \end{tips}
    }
    
\newcommand{\introTips}
    {
    \begin{tips}
        $<$ 1 page, approx. 3 paragraphs \medskip

        \noindent Begin by framing the broad field in which you are working very briefly. You do not need to provide a full review of this field. Strategically cite a few reviews. Just identify the main relevant work and cite review articles for those who need more context. Get to the critical gap/problem as quickly as possible. 
        
        Then elaborate on the specific problem that you are working on and explain why solving it is important (if you haven’t already). Presumably, other folks have also looked at this problem. Briefly review what they have found and/or the relevant hypotheses. Set up why there is a gap/problem (e.g., \textit{We need to test this new hypothesis, there are better machine learning methods, etc.}).
        
        Then explicitly detail what questions/hypotheses/etc you are trying to answer/test. Feel free to use a list. Describe your innovative approach---you may introduce such an approach in the paragraph(s) directly before this one. Note major predictions here and reemphasize the significance. Someone should be able to read only this paragraph and have a good sense of what the study/you/the team has tried to accomplish. (You can bullet point or list out the aims/problems/questions if you really want to draw attention to them).
    \end{tips}
    }

\newcommand{\resultsTips}
    {
    \begin{tips}
        As long as needed, but usually 4-7 subsections \medskip
        
        \noindent Divide this section into subsections with a logical progression from one analysis to another. The subsection titles should provide an outline of the results that enables skimming. Thus, try to make each results subsection title state a brief result. Do not feel constrained to follow the order in which you did the analyses. Tell the most logical and easy to follow story that highlights the most exciting parts of your findings early and ends with a finding that leads most towards future directions (or is a preliminary finding toward the new direction).  Make sure to reference and use all data presented in figures and tables. Negative results are useful. Actions should be written in past-tense, while statements/results are in the present-tense. E.g., (\textit{We TESTED the correlation of X vs. Y, and we OBSERVED that they ARE correlated}).
        
        Each results section should contain the following: context, approach, result, details, conclusion (see below for template). Repeat as needed. Depending on the relatedness of results, it is possible to have multiple of these ‘modules’ under a single heading. Conversely, you may have more headings than figures if, for example, multiple sections are needed to describe a multi-panel figure. (In this case, consider if the figure should be split up or not.)
    \end{tips}
    }
    
\newcommand{\captionTips}
    {
    \begin{tips}
        \noindent Tips for captions and figures: All figures and tables should be inserted into the document as soon as possible after the first mention in the text. Make sure to number them sequentially as they are referenced in the text and provide a succinct descriptive caption. If there are multiple parts to a figure, give clear capital letter labels (A, B, C, etc.) to each panel. Make sure each figure and sub-figure are referenced somewhere in the text! The images should be saved in vector formats (such as PDF or EPS) to maintain high resolution. The exceptions to this are: 1) if you are including a photograph or 2) if you are plotting a file with thousands of overlapping individual data points that would each be represented in the vector file (use a bitmap format here). The title of the figure caption should state the overarching result. (This will often be similar to a results section title.) Captions should interpret the results briefly. The figures and captions should be able to be ``stand alone'' from the main text with enough context and methods for interpretation. (Many readers will only look at the figures and captions.) Subplots should contribute to one major overarching scientific point. If you can’t come up with one conclusion/title for a figure, consider if it should be two figures. If two figures show the same conclusion, one should probably go to the supplement (e.g., demonstrating that a results replicates across cell types or with a different control).

    \end{tips}
    }
    
\newcommand{\discussionTips}
    {
    \begin{tips}
        Approx 1 page \medskip
        
        \noindent The discussion should be a broad overview of the significance of your findings to the community you are addressing. You can mirror the results section, but you don’t have to. You should start with a brief summary of the main findings and then a paragraph for the accompanying context and interpretation of each of the big picture conclusions or contributions. Then you should address high-level limitations of the study (not just small limitations that you might address in your results/methods) Finally, end with future directions on a positive note. How can people use your model/framework results? This positive-critical-positive is a compliment sandwich.

        Everything discussed should be within the frame of reference of your paper’s major conclusions or contributions. You will likely want to address a shortlist of the key papers your paper speaks to, but this is not a literature review. Even if you don’t talk extensively about a paper, you can still cite it if it is relevant to your conclusions. (There is little cost to being free with your citations!) 
        
        Contrary to popular belief, you can bring up a new result in the discussion, especially if it is preliminary or a response to reviewer critiques. Save this for findings that are in service to the discussion, rather than just a main finding of a paper. You can also use sub-sections if it is helpful in framing but this is somewhat non-standard. 

        Optional: Consider proposing a model or framework in your discussion that integrates or contextualizes your results. This can be accompanied by a schematic or ``model'' figure. This is a great way to emphasize the contribution of your study and guide future work.

    \end{tips}
    }
    
\newcommand{\methodsTips}
    {
    \begin{tips}
        No length limit \medskip
        
        \noindent \underline{Goal}: Communicate the data and methods used to produce the results. Start writing this as you are doing your analyses. \underline{Objective}: (1) To write clearly how you produced all parts of the results and (2) promote reproducibility of the findings. \underline{Style}: Overall, the methods section should be clearly and thoroughly written. In theory, any knowledgeable reader (e.g., a graduate student working in your area) would be able to reproduce any of the findings just by reading your methods. \underline{Organization}: Write subsections with clear subheadings for each method. These should be descriptive. Ideally, subsections will be structured so that the reader can “look up” the details for any subsection of the results. \underline{Ordering subsections}: Subsection order should generally match the order of the results. However, if you repeat the same type of analyses in two separate parts of the paper, you can describe this once and refer to the sections those methods address. \underline{Voice}: Write in the active past tense (E.g. “We computed…, we intersected…, we quantified…”). \underline{Note on publicly available datasets}: Include a citation or url link to datasets used, along with the last download date. \underline{Citations}: if you use it, cite it. There is no ambiguity here.
        
        In some journals, the methods section precedes the results section, while in others it follows the Discussion section. If it comes before, then the text should provide some orientation and motivation. If it comes after, then it can be more of a list. Unless you know it will come first, write the methods section as if it were following the discussion section and then add orienting text later if necessary. 
        
        \underline{Other tips}: Think of the methods section is a cookbook, and each subsection is its own chapter. Some chapters will discuss where to get the finest ingredients, some will discuss specific cooking techniques, and the rest will discuss the recipes that combine ingredients and technique. For each recipe (subsection), you must be clear on which ingredients are needed and the order that the ingredients are assembled.
        
        You do not need to divulge every quotidian analytical detail (e.g., \textit{I first loaded .tsv into a data frame using pandas (v14.0) function pd.read\_csv(filename) and pivoted it into a table}). But you should provide every analytical detail relevant to reproducing the final data analyzed (e.g. \textit{We removed all loci mapping to sex chromosomes using the subtract function from BEDTools (v2.2.1; Quinlan and Hall 2010).})
    \end{tips}
    }
    
\newcommand{\methodsDataTips}
    {
    \begin{tips}
        \begin{itemize}[noitemsep]
            \item For publicly available data: Dataset name, genome build, url/citation, last-download date, sample size, etc.
            \item For private data: Describe samples, sample collection, IRBs, names of kits and reagents used to process samples, sample size. 
            \item Include any details about the dataset critical for analyses and generation of results (I.e. controls, selection criteria, covariates, ancestry, age, sex, status, cell type, tissue source, assay details). 
            \item If applicable, explain how data were processed (i.e. excluded sex chromosomes, removed centromeric and repeat elements, etc.) 
            \item Ask someone else in the lab to look over the draft to make sure you have not left out any essential details.
        \end{itemize}
    \end{tips}
    }
    
\newcommand{\methodsAnalysisTips}{
    \begin{tips}
        \begin{itemize}[noitemsep]
            \item State the data inputs. 
            \item State your controls.  
            \item State sample sizes.
            \item State any software package, its version, and arguments used to run the analysis. 
            \item Explain any processing steps, the order, and the rationale of those steps relevant for producing the result figure. 
        \end{itemize}
    \end{tips}
    }
    
\newcommand{\generalTips}
    {
    \begin{tips}
        \begin{itemize}[noitemsep]
            \item Formatting/scientific corrections 
                \begin{itemize}[noitemsep]
                    \item The first time you report a p-value, give the test used. 
                    \item Report p values as ($P = 0.\#$), be careful with ($P = 0$) (use $< \#$ when $P$ is very small)
                    \item Change “mutation” to “variant”...maybe change “SNP” or “SNV” to “variant(s)” if specificity is not important (i.e., don’t forget that indels and CNVs/SVs are also types of genetic variants)
                    \item For genomic coordinates/distances, the format is \# kb or \# Mb (space between number and unit, not \# Kb or \# mb)
                    \item Use American spelling conventions.
                \end{itemize}
            \item Organization
                \begin{itemize}[noitemsep]
                    \item Put answers to the “why” question in the discussion
                    \item Bring important findings to the first sentence of a paragraph/section
                    \item With section headers, figure titles, and paper titles, try to state a result if possible (rather than a method, e.g. “Neanderthals were bald” rather than “Investigating hair patterns of Neanderthals”)
                \end{itemize}
            \item Wordsmithing
                \begin{itemize}[noitemsep]
                    \item Where you can avoid being vague, be specific:  instead of “establishes the relationship” say the relationship that was established. Instead of “Is incompletely understood” say what is the specific thing that is incompletely understood
                    \item Get rid of “metadiscourse”. E.g., Change “Our findings XY and Z support previous findings, including studies by Smith et al  and Johnson et al, which reported that Neanderthals might have been bald at one point” to “XY and Z support findings that Neanderthals may have been bald (citation)”.
                    \item Don’t claim you’re the first ever (or save that for the cover later)
                    \item Generally speaking, avoid passive voice. If you can add “by zombies” after the verb and it still makes sense...rethink your phrasing
                \end{itemize}
            \item Word choice
                \begin{itemize}[noitemsep]
                    \item Change utilize to use 
                    \item Change Impacted to influenced
                    \item Delete uses of “those/these” in reference to previous sentence (make sure the subject of the sentence is clear)
                    \item “Characterize” connotes that you don’t really have a hypothesis (try quantify?)
                    \item Some favorite transition words: thus, nonetheless, furthermore, indeed, therefore
                    \item Fewer (discrete quantities) vs less (continuous quantities)
                    \item (e.g.,\dots) and (i.e.,\dots) can be helpful
                    \item “Data” is the plural of “datum”.
                    \item Avoid personifying inanimate objects (“the gene wants to...”)
                    \item Splitting infinitives is ok.
                    \item Contractions are not acceptable in academic writing.
                    \item Remove: Interestingly, surprisingly
                \end{itemize}
            \item Punctuation
                \begin{itemize}[noitemsep]
                    \item Parenthetical punctuation: clauses and sentences
                    \item Use an Oxford comma to separate the last and second-to-last elements in a list.
                    \item Beware of consistency between Hyphens(-), n-dash(--), m-dash(---)
                    \item With reporting intervals be consistent in how you use dashes and spaces (1-2 vs 1 - 2 vs 1 – 2, etc)
                    \item Use a comma between independent clauses joined by a conjunction.
                    \item Use "that" for restrictive clauses (no commas). Use "which" for non-restrictive clauses (use commas). 
                    \item Pluralization rules for abbreviations.
                \end{itemize}
        \end{itemize}
    \end{tips}
    }
    
). Then add your own content to the relevant subfiles.

% Figures should be uploaded to the ./main_figs and ./supplement/suppl_figs
% folders. Tables should be uploaded to ./supplement/suppl_tabs. When you work
% with main text figures, you should create a "code name" for them. You will
% follow the format in 4maintextFigs.tex to add a function that will generate
% the figure. When you want that figure to appear in the manuscript call that
% function (e.g. \codename).

% Note that this document contains info about how to customize these things:
% fontsize, font (to Arial), line spacing, line numbers, left vs. justified
% aligned, number or author-date citations, natbib vs bibLaTeX citations,
% sideways tables and figures, figure references and more!

% See these resources for more LaTeX tips:
%https://en.m.wikibooks.org/wiki/LaTeX
%https://tobi.oetiker.ch/lshort/lshort.pdf
%https://gangw.cs.illinois.edu/latex.pdf
%https://web.mit.edu/rsi/www/pdfs/new-latex.pdf
%http://hoffman.physics.harvard.edu/Hoffman-example-paper.pdf

%%% ------------ READ IN PACKAGES & SETUP ------------ %%%

% Basics, font size, page setup, link format, line spacing/numbering
\documentclass[11pt]{article} % font size
\usepackage[utf8]{inputenc}
\usepackage[english]{babel}
\usepackage[a4paper,margin=1in]{geometry} % page setup, margins
\usepackage{authblk} % author affiliations
\usepackage[colorlinks=true, linkcolor=black, citecolor=black, urlcolor=blue, filecolor=blue,breaklinks=true]{hyperref} % Hyperlinks, needed for biblatex, other option: allcolors=blue
\usepackage{setspace} % for linespacing
\usepackage{lineno} % line numbering (change with \linespacing below)
\usepackage{csquotes} % Recommended for biblatex, must load after lineno or will get a warning
\usepackage{ragged2e} % for left aligning
\setlength{\RaggedRightParindent}{\parindent} % for paragraph indent with left aligning

% Figure references and bolding them https://tex.stackexchange.com/questions/87903/bold-cross-references
\usepackage[capitalise, noabbrev, nameinlink]{cleveref}
\crefdefaultlabelformat{#2\textbf{#1}#3} % To bold figure refs
\Crefname{figure}{\textbf{Figure}}{\textbf{Figures}} % To bold figure refs
\Crefname{section}{\textbf{Section}}{\textbf{Sections}} % To bold section refs
\Crefname{table}{\textbf{Table}}{\textbf{Tables}} % To bold table refs
\newcommand{\crefrangeconjunction}{--} % change conjunction b/w figures
\newcommand{\crefpairconjunction}{, } % change conjunction b/w figures
\newcommand{\creflastconjunction}{, } % change conjunction b/w figures
\newcommand{\crefmiddleconjunction}{, } % change conjunction b/w figures

% Figures, Tables and captions
\usepackage{graphicx}
\graphicspath{{./main_figs/}{./supplement/suppl_figs/}} % paths for figs
\DeclareGraphicsExtensions{.pdf,.jpeg,.JPG,.png,.PNG, .eps, .tiff}
\usepackage{subcaption} % for subcaptions (panels), and add parentheses
\DeclareCaptionLabelFormat{bold}{\textbf{(#2)}} % bold subpanel letter in caption
\captionsetup{subrefformat=bold} % bold subpanel letter in caption
\renewcommand{\thesubfigure}{\Alph{subfigure}} % Make subpanel numbering capitalized
\newcommand{\labelphantom}[1]{%  To make subpanel references easier from https://tex.stackexchange.com/a/255790/121424 
  \parbox{0pt}{\phantomsubcaption\label{#1}}%
}
\usepackage{pgffor} % for function to make figures with subpanels
\usepackage{alphalph} % for function to make figures with subpanels
\usepackage[labelfont=bf, textfont=bf, singlelinecheck=off, textfont=footnotesize]{caption} % boldness of captions
\usepackage{booktabs}
\usepackage{multirow}
\usepackage{rotating}
%\usepackage{tabularx}
%\usepackage{makecell}

% Read in helper functions
% Functions to create figures
% \generateFigSubpanels: Figure with subpanels
% \generateFig: Figure without subpanels
% \generateSidewaysFigSubpanels: Sideways fig and caption with subpanels
% \generateSidewaysFig: Sideways fig and caption without subpanels
% \generateTab: Table
% \generateSidewaysTab: Sideways table and caption

%%%% Function to create and format figures with subpanels %%%%
\newcommand{\generateFigSubpanels}[6][0]{% default rotation=0
     \expandafter\newcommand\csname#2\endcsname{ %https://tex.stackexchange.com/questions/65780/macro-defining-macro/65781#65781
      \begin{figure}[htbp]
        \foreach [count=\i] \x in #6{%
            \labelphantom{fig:#2:\AlphAlph{\i}}
        }
        \centering
        \includegraphics[width=#4\linewidth,angle=#1]{#3}
        \caption{\textbf{\normalsize #5}}\label{fig:#2}
        \footnotesize
        \justifying
        \foreach [count=\i] \x in #6{%
            \noindent\subref{fig:#2:\AlphAlph{\i}}~\x\space
        }
      \end{figure}
    }
}

% USAGE:
% \generateFigSubpanels[optionalRotationDegrees,default=0]{figCodeName}{figureNamePath}{size:propOfTextwidth}
%     {main caption}
%     {{
%         {subcaption1},
%         {subcaption2},
%         {subcaption3}
%     }}

% Then in the reference your figures with \cref{fig:figCodeName}
% Reference subpanels with \cref{fig:figCodeName:A}


%%%% Function to create and format figures without subpanels %%%%
\newcommand{\generateFig}[6][0]{%  default rotation=0
     \expandafter\newcommand\csname#2\endcsname{ %https://tex.stackexchange.com/questions/65780/macro-defining-macro/65781#65781
      \begin{figure}[htbp]
        \centering
        \includegraphics[width=#4\linewidth,angle=#1]{#3}
        \caption{\textbf{\normalsize #5}\label{fig:#2}
        \footnotesize
        \normalfont
        #6
        }
        
      \end{figure}
    }
}

% USAGE:

% \generateFig[optionalRotationDegrees,default=0]{figCodeName}{figureNamePath}{size:propOfTextwidth}
%     {Bold text main caption}
%     {Smaller normal text caption}

% Then in the main text reference your figures with \cref{fig:figCodeName}

% Versions to turn caption sideways

% To turn caption and figure sideways
\newcommand{\generateSidewaysFigSubpanels}[6][0]{% default rotation=0
     \expandafter\newcommand\csname#2\endcsname{ %https://tex.stackexchange.com/questions/65780/macro-defining-macro/65781#65781
      \begin{sidewaysfigure}[htbp]
        \foreach [count=\i] \x in #6{%
            \labelphantom{fig:#2:\AlphAlph{\i}}
        }
        \centering
        \includegraphics[width=#4\linewidth,angle=#1]{#3}
        \caption{\textbf{\normalsize #5}}\label{fig:#2}
        \footnotesize
        \justifying
        \foreach [count=\i] \x in #6{%
            \noindent\subref{fig:#2:\AlphAlph{\i}}~\x\space
        }
      \end{sidewaysfigure}
    }
}

\newcommand{\generateSidewaysFig}[6][0]{%  default rotation=0
     \expandafter\newcommand\csname#2\endcsname{ %https://tex.stackexchange.com/questions/65780/macro-defining-macro/65781#65781
      \begin{sidewaysfigure}[htbp]
        \centering
        \includegraphics[width=#4\linewidth,angle=#1]{#3}
        \caption{\textbf{\normalsize #5}\label{fig:#2}
        \footnotesize
        \normalfont
        #6
        }
        
      \end{sidewaysfigure}
    }
}

%%%% Function to create and format Tables %%%%

\newcommand{\generateTab}[6][]{%  Note that rotation does not work here
     \expandafter\newcommand\csname#2\endcsname{ \begin{table}[ht]
          \centering
                \resizebox{#4\textwidth}{!}{
                \input{#3}
            }
            \caption{\textbf{\normalsize #5}\label{tab:#2}
            \footnotesize
            \normalfont
            #6
            }
            
        \end{table}
    }
}

% Sideways table
\newcommand{\generateSidewaysTab}[6][]{%  Note that rotation does not work here
     \expandafter\newcommand\csname#2\endcsname{ \begin{sidewaystable}[ht]
          \centering
                \resizebox{#4\textwidth}{!}{
                \input{#3}
            }
            \caption{\textbf{\normalsize #5}\label{tab:#2}
            \footnotesize
            \normalfont
            #6
            }
            
        \end{sidewaystable}
    }
}


% Change Fonts
%\usepackage{fontspec} % To use arial font but you must change your compiler to XeLaTeX (upper left corner Menu > compilier) 
%\setmainfont{Arial} % To use arial font but you must change your compiler to XeLaTeX (upper left corner Menu > compilier) 

% Can remove, are for the tips
\usepackage[dvipsnames]{xcolor} % can remove, for coloring the tips
\usepackage{changepage} % can remove, for width on the tips
\usepackage{enumitem} % can remove, for width on the tips


%%% ------------ REFERENCES FORMAT ------------ %%% 
% Note: Change the print bibliography command if switching between BibLaTex and Natbib
% See this for a great comparison: https://tex.stackexchange.com/questions/25701/bibtex-vs-biber-and-biblatex-vs-natbib

%% BibLaTeX format %% (newer/better but not all journals accept)
% Numbered:
%\usepackage[sorting=none, backend=biber, style=numeric-comp,maxbibnames=10,minbibnames=3, maxcitenames= 2,giveninits=true, natbib=true, url=false,hyperref=true]{biblatex}
% Author Year:
\usepackage[backend=biber,style=authoryear-comp,sorting=nyt,natbib=true, url=false]{biblatex}

\addbibresource{library.bib} % add bibliography resource

%% Natbib format %% (more basic)
% Numbered:
% \usepackage[square,numbers]{natbib}
% \bibliographystyle{unsrtnat}
% Author Year:
%\usepackage{natbib}
%\bibliographystyle{unsrtnat}


%%% ------------ DOCUMENT ORGANIZATION ------------ %%%

%%% TITLE, AUTHOR, AFFILIATIONS, ABSTRACT %%%

%%% ------------ TITLE ------------ %%%

\title{The Capra Lab manuscript template\titleTips}

%%% ------------ AUTHORS ------------ %%%

\author[1]{Your~Name~Here}
\author[2,3,*]{John~A.~Capra}
\affil[1]{Place, City, XX}
\affil[2]{Bakar Computational Health Sciences Institute, University of California, San Francisco, CA}
\affil[3]{Department of Epidemiology and Biostatistics, University of California, San Francisco, CA}

\affil[ ]{ } % for some blank space
\affil[*]{\textit{Correspondence to }\underline{tony@capralab.org}}
\renewcommand\Affilfont{\footnotesize} % size of affilitations
\setcounter{Maxaffil}{0} % no max for num affiliations
\date{} % If you want a date fill this in (e.g. \date{February 2022})

%%% ------------ ABSTRACT ------------ %%%

\newcommand{\makeAbstract}{
\begin{abstract}
\abstractTips

\paragraph{Background}
\noindent (No more than three sentences.) Motivate your work by stating the problem you addressed or the hypothesis you tested. The last sentence should clearly state the gap in knowledge.

\paragraph{Methods}
(One or two optional sentences; this can be combined with results if it is not a methods focused paper. \textit{To address this gap, we applied X to Y.} or \textit{we synthesized V and W to reveal Z.}

\paragraph{Results}
(Most of the rest of the abstract) \textit{Here, we show\dots} What did you do and what did you find? This should summarize the main analyses and results of your work. Keep it high level; it is ok if some results are not mentioned.

\paragraph{Conclusions}
(No more than two sentences.) What did you learn from your studies? How does this change the field? What do your results say about the problem/hypothesis identified in the Background section?
\end{abstract}

}


\begin{document}

    %\RaggedRight % If you want left-aligned (ie ragged right text) rather than justified
    \setstretch{1.15} %\onehalfspace
    \linenumbers
    %%%% Delete this file to remove all the tips %%%%
% Make sure you delete all the \[section]Tips in the main text files!!

\newenvironment{tips} % To make the blue/gray-colored, small text environment to format the tips
    {
        \footnotesize\color{CadetBlue}
        \begin{adjustwidth}{1cm}{1cm}
        \begin{singlespace}
        \vspace{.5em}
        \nolinenumbers
    }
    {
        \end{singlespace}
        \end{adjustwidth}
        \vspace{.5em}
        %\ignorespacesafterend
    }
    

    
\newcommand{\titleTips}
    {
    \begin{tips}
    \centering
        State a result with the title if possible. Aim for fewer than 10 words.
    \end{tips}
    }
    
\newcommand{\abstractTips}
    {
    \begin{tips}
        \noindent $<$ 250 words. Shorter is better. Check journal limits, structured or unstructured, and so on. \medskip
        
        \noindent Make a compelling elevator pitch for the paper that includes a brief intro, methods, results, conclusions. It can be split into sections or not. Always start with sections to provide a framework, but remove if needed. Revisit the abstract as you write the paper (even if you have written abstracts for conferences previously), because you will find better ways to summarize the paper as it evolves! 
        
        First sentence should make a broad and general statement that sets up to the importance of the topic. Avoid definitions in the first sentence. For example, don’t say \textit{TADs are 3D structures.} It is boring and does not convey importance or a gap. The first sentence can be the same as the first sentence of the Intro but often is more specific/less broad. Overall, it should establish the gap in knowledge as briefly as possible and avoid too much background. Put your main question/gap early (2\textsuperscript{nd} or 3\textsuperscript{rd} sentence). End the abstract by making the significance and implications for the field (and maybe next steps) clear. Be as broad as possible.  No citations and minimize explicit references to other work in the abstract (meta-discourse like \textit{Previous work that investigated X showed Y}); instead, just talk about Y and the gap that you will fill.
    \end{tips}
    }
    
\newcommand{\introTips}
    {
    \begin{tips}
        $<$ 1 page, approx. 3 paragraphs \medskip

        \noindent Begin by framing the broad field in which you are working very briefly. You do not need to provide a full review of this field. Strategically cite a few reviews. Just identify the main relevant work and cite review articles for those who need more context. Get to the critical gap/problem as quickly as possible. 
        
        Then elaborate on the specific problem that you are working on and explain why solving it is important (if you haven’t already). Presumably, other folks have also looked at this problem. Briefly review what they have found and/or the relevant hypotheses. Set up why there is a gap/problem (e.g., \textit{We need to test this new hypothesis, there are better machine learning methods, etc.}).
        
        Then explicitly detail what questions/hypotheses/etc you are trying to answer/test. Feel free to use a list. Describe your innovative approach---you may introduce such an approach in the paragraph(s) directly before this one. Note major predictions here and reemphasize the significance. Someone should be able to read only this paragraph and have a good sense of what the study/you/the team has tried to accomplish. (You can bullet point or list out the aims/problems/questions if you really want to draw attention to them).
    \end{tips}
    }

\newcommand{\resultsTips}
    {
    \begin{tips}
        As long as needed, but usually 4-7 subsections \medskip
        
        \noindent Divide this section into subsections with a logical progression from one analysis to another. The subsection titles should provide an outline of the results that enables skimming. Thus, try to make each results subsection title state a brief result. Do not feel constrained to follow the order in which you did the analyses. Tell the most logical and easy to follow story that highlights the most exciting parts of your findings early and ends with a finding that leads most towards future directions (or is a preliminary finding toward the new direction).  Make sure to reference and use all data presented in figures and tables. Negative results are useful. Actions should be written in past-tense, while statements/results are in the present-tense. E.g., (\textit{We TESTED the correlation of X vs. Y, and we OBSERVED that they ARE correlated}).
        
        Each results section should contain the following: context, approach, result, details, conclusion (see below for template). Repeat as needed. Depending on the relatedness of results, it is possible to have multiple of these ‘modules’ under a single heading. Conversely, you may have more headings than figures if, for example, multiple sections are needed to describe a multi-panel figure. (In this case, consider if the figure should be split up or not.)
    \end{tips}
    }
    
\newcommand{\captionTips}
    {
    \begin{tips}
        \noindent Tips for captions and figures: All figures and tables should be inserted into the document as soon as possible after the first mention in the text. Make sure to number them sequentially as they are referenced in the text and provide a succinct descriptive caption. If there are multiple parts to a figure, give clear capital letter labels (A, B, C, etc.) to each panel. Make sure each figure and sub-figure are referenced somewhere in the text! The images should be saved in vector formats (such as PDF or EPS) to maintain high resolution. The exceptions to this are: 1) if you are including a photograph or 2) if you are plotting a file with thousands of overlapping individual data points that would each be represented in the vector file (use a bitmap format here). The title of the figure caption should state the overarching result. (This will often be similar to a results section title.) Captions should interpret the results briefly. The figures and captions should be able to be ``stand alone'' from the main text with enough context and methods for interpretation. (Many readers will only look at the figures and captions.) Subplots should contribute to one major overarching scientific point. If you can’t come up with one conclusion/title for a figure, consider if it should be two figures. If two figures show the same conclusion, one should probably go to the supplement (e.g., demonstrating that a results replicates across cell types or with a different control).

    \end{tips}
    }
    
\newcommand{\discussionTips}
    {
    \begin{tips}
        Approx 1 page \medskip
        
        \noindent The discussion should be a broad overview of the significance of your findings to the community you are addressing. You can mirror the results section, but you don’t have to. You should start with a brief summary of the main findings and then a paragraph for the accompanying context and interpretation of each of the big picture conclusions or contributions. Then you should address high-level limitations of the study (not just small limitations that you might address in your results/methods) Finally, end with future directions on a positive note. How can people use your model/framework results? This positive-critical-positive is a compliment sandwich.

        Everything discussed should be within the frame of reference of your paper’s major conclusions or contributions. You will likely want to address a shortlist of the key papers your paper speaks to, but this is not a literature review. Even if you don’t talk extensively about a paper, you can still cite it if it is relevant to your conclusions. (There is little cost to being free with your citations!) 
        
        Contrary to popular belief, you can bring up a new result in the discussion, especially if it is preliminary or a response to reviewer critiques. Save this for findings that are in service to the discussion, rather than just a main finding of a paper. You can also use sub-sections if it is helpful in framing but this is somewhat non-standard. 

        Optional: Consider proposing a model or framework in your discussion that integrates or contextualizes your results. This can be accompanied by a schematic or ``model'' figure. This is a great way to emphasize the contribution of your study and guide future work.

    \end{tips}
    }
    
\newcommand{\methodsTips}
    {
    \begin{tips}
        No length limit \medskip
        
        \noindent \underline{Goal}: Communicate the data and methods used to produce the results. Start writing this as you are doing your analyses. \underline{Objective}: (1) To write clearly how you produced all parts of the results and (2) promote reproducibility of the findings. \underline{Style}: Overall, the methods section should be clearly and thoroughly written. In theory, any knowledgeable reader (e.g., a graduate student working in your area) would be able to reproduce any of the findings just by reading your methods. \underline{Organization}: Write subsections with clear subheadings for each method. These should be descriptive. Ideally, subsections will be structured so that the reader can “look up” the details for any subsection of the results. \underline{Ordering subsections}: Subsection order should generally match the order of the results. However, if you repeat the same type of analyses in two separate parts of the paper, you can describe this once and refer to the sections those methods address. \underline{Voice}: Write in the active past tense (E.g. “We computed…, we intersected…, we quantified…”). \underline{Note on publicly available datasets}: Include a citation or url link to datasets used, along with the last download date. \underline{Citations}: if you use it, cite it. There is no ambiguity here.
        
        In some journals, the methods section precedes the results section, while in others it follows the Discussion section. If it comes before, then the text should provide some orientation and motivation. If it comes after, then it can be more of a list. Unless you know it will come first, write the methods section as if it were following the discussion section and then add orienting text later if necessary. 
        
        \underline{Other tips}: Think of the methods section is a cookbook, and each subsection is its own chapter. Some chapters will discuss where to get the finest ingredients, some will discuss specific cooking techniques, and the rest will discuss the recipes that combine ingredients and technique. For each recipe (subsection), you must be clear on which ingredients are needed and the order that the ingredients are assembled.
        
        You do not need to divulge every quotidian analytical detail (e.g., \textit{I first loaded .tsv into a data frame using pandas (v14.0) function pd.read\_csv(filename) and pivoted it into a table}). But you should provide every analytical detail relevant to reproducing the final data analyzed (e.g. \textit{We removed all loci mapping to sex chromosomes using the subtract function from BEDTools (v2.2.1; Quinlan and Hall 2010).})
    \end{tips}
    }
    
\newcommand{\methodsDataTips}
    {
    \begin{tips}
        \begin{itemize}[noitemsep]
            \item For publicly available data: Dataset name, genome build, url/citation, last-download date, sample size, etc.
            \item For private data: Describe samples, sample collection, IRBs, names of kits and reagents used to process samples, sample size. 
            \item Include any details about the dataset critical for analyses and generation of results (I.e. controls, selection criteria, covariates, ancestry, age, sex, status, cell type, tissue source, assay details). 
            \item If applicable, explain how data were processed (i.e. excluded sex chromosomes, removed centromeric and repeat elements, etc.) 
            \item Ask someone else in the lab to look over the draft to make sure you have not left out any essential details.
        \end{itemize}
    \end{tips}
    }
    
\newcommand{\methodsAnalysisTips}{
    \begin{tips}
        \begin{itemize}[noitemsep]
            \item State the data inputs. 
            \item State your controls.  
            \item State sample sizes.
            \item State any software package, its version, and arguments used to run the analysis. 
            \item Explain any processing steps, the order, and the rationale of those steps relevant for producing the result figure. 
        \end{itemize}
    \end{tips}
    }
    
\newcommand{\generalTips}
    {
    \begin{tips}
        \begin{itemize}[noitemsep]
            \item Formatting/scientific corrections 
                \begin{itemize}[noitemsep]
                    \item The first time you report a p-value, give the test used. 
                    \item Report p values as ($P = 0.\#$), be careful with ($P = 0$) (use $< \#$ when $P$ is very small)
                    \item Change “mutation” to “variant”...maybe change “SNP” or “SNV” to “variant(s)” if specificity is not important (i.e., don’t forget that indels and CNVs/SVs are also types of genetic variants)
                    \item For genomic coordinates/distances, the format is \# kb or \# Mb (space between number and unit, not \# Kb or \# mb)
                    \item Use American spelling conventions.
                \end{itemize}
            \item Organization
                \begin{itemize}[noitemsep]
                    \item Put answers to the “why” question in the discussion
                    \item Bring important findings to the first sentence of a paragraph/section
                    \item With section headers, figure titles, and paper titles, try to state a result if possible (rather than a method, e.g. “Neanderthals were bald” rather than “Investigating hair patterns of Neanderthals”)
                \end{itemize}
            \item Wordsmithing
                \begin{itemize}[noitemsep]
                    \item Where you can avoid being vague, be specific:  instead of “establishes the relationship” say the relationship that was established. Instead of “Is incompletely understood” say what is the specific thing that is incompletely understood
                    \item Get rid of “metadiscourse”. E.g., Change “Our findings XY and Z support previous findings, including studies by Smith et al  and Johnson et al, which reported that Neanderthals might have been bald at one point” to “XY and Z support findings that Neanderthals may have been bald (citation)”.
                    \item Don’t claim you’re the first ever (or save that for the cover later)
                    \item Generally speaking, avoid passive voice. If you can add “by zombies” after the verb and it still makes sense...rethink your phrasing
                \end{itemize}
            \item Word choice
                \begin{itemize}[noitemsep]
                    \item Change utilize to use 
                    \item Change Impacted to influenced
                    \item Delete uses of “those/these” in reference to previous sentence (make sure the subject of the sentence is clear)
                    \item “Characterize” connotes that you don’t really have a hypothesis (try quantify?)
                    \item Some favorite transition words: thus, nonetheless, furthermore, indeed, therefore
                    \item Fewer (discrete quantities) vs less (continuous quantities)
                    \item (e.g.,\dots) and (i.e.,\dots) can be helpful
                    \item “Data” is the plural of “datum”.
                    \item Avoid personifying inanimate objects (“the gene wants to...”)
                    \item Splitting infinitives is ok.
                    \item Contractions are not acceptable in academic writing.
                    \item Remove: Interestingly, surprisingly
                \end{itemize}
            \item Punctuation
                \begin{itemize}[noitemsep]
                    \item Parenthetical punctuation: clauses and sentences
                    \item Use an Oxford comma to separate the last and second-to-last elements in a list.
                    \item Beware of consistency between Hyphens(-), n-dash(--), m-dash(---)
                    \item With reporting intervals be consistent in how you use dashes and spaces (1-2 vs 1 - 2 vs 1 – 2, etc)
                    \item Use a comma between independent clauses joined by a conjunction.
                    \item Use "that" for restrictive clauses (no commas). Use "which" for non-restrictive clauses (use commas). 
                    \item Pluralization rules for abbreviations.
                \end{itemize}
        \end{itemize}
    \end{tips}
    }
    
 % For the tips, eventually remove!
    
    \maketitle
    \makeAbstract
    \clearpage 
    \section{INTRODUCTION}
\introTips % Delete when no longer needed
Introductions for Cell Press journal articles  should not include any subsections. A quick note about citations: they should occur in the form of superscript numbers, numbered in order of appearance\supercite{enardIntraInterspecificVariation2002,kingEvolutionTwoLevels1975,sholtisGeneRegulationOrigins2010}. Use the \texttt{supercite} command immediately after the relevant text to write citations using the citation keys from the .bib file. You can also explicitly refer to the author(s)/paper using the following convention: ``Enard et al.\supercite{enardIntraInterspecificVariation2002} quantified gene expression in four primate species''. Cell Press Journals use journal abbreviations in the References section. If you export a .bib file from a reference manager (e.g., Mendeley, Zotero), ensure that journal abbreviations are invoked. Abbreviated components of the journal title should also include a period (see the Sholtis et al. 2010 reference below, Trends in Genetics $\rightarrow$ Trends Genet.). These can be added manually in your reference management software library or code can be added in the 0main.tex file to do this automatically.

\textbf{First paragraph(s)} What is the problem? Why do we care?  First sentence can be a general truth or general limitation that your audience accepts. E.g., \textit{While differences between archaic and modern humans are well described, there is a poor understanding of the genetic mechanisms underlying such phenotypic differences.} or  \textit{Most aspects of archaic hominin biology cannot be directly studied due to their lack of preservation in fossils.} Don't waste the first sentence with a definition.

\textbf{Middle paragraph(s)} What have others found about this specific problem? What are the current hypotheses and support for each? and/or what data have been previously analyzed? What are the gaps? E.g., \textit{One such mechanism may be alternative splicing, but we cannot obtain such data from extinct taxa.} This part of the Intro may require a few paragraphs, but you do not need to fully review the history of work on the question. Focus on the current state of the field.

\textbf{Last paragraph} Overview of the major questions addressed here and why they are important. \textit{Here, we investigated X using dataset Y and dataset Z. We tested the A hypothesis. We also considered this other obscure hypothesis. We predicted x, y, and z.} E.g., \textit{Here, we leverage a new deep neural network that can predict splice altering variants from sequence alone to examine such variants in archaic hominins.} 
    \clearpage % can remove if you don't want a new page 
    % Create a "code name" for each figure. In the maintext, when you want that figure to appear in the manuscript call that function (e.g. \figCodeName).

%%%% Create and format Figure with subpanels: %%%%
% \generateFigSubpanels[optionalRotationDegrees]{figCodeName}{figureNamePath}{size:propOfTextwidth}
%     {main caption}
%     {{
%         {subcaption1},
%         {subcaption2},
%         {subcaption3}
%     }}

% Then in the reference your figures with \cref{fig:figCodeName} and Reference subpanels with \cref{fig:figCodeName:A}


%%%% Function to create and format figures without subpanels %%%%

% \generateFig[optionalRotationDegrees]{figCodeName}{figureNamePath}{size:propOfTextwidth}
%     {Bold text main caption}
%     {Smaller normal text caption}

% Then in the main text reference your figures with \cref{fig:figCodeName}

% You can also use \generateSidwaysFigSubpanels or \generateSidewaysFig to rotate the caption with the figure.

%%%% Example figure with codename: figureSuggestions %%%%

\generateFigSubpanels{figureSuggestions}{figureSuggestions.pdf}{1}
    {Figure title which should not end in a period} % Main caption
    {{
        {Succinct caption for Panel A. Note that figure captions will not split across pages.}, % Subcaption 1
        {Caption for panel B. Add additional subpanel captions as needed depending on the figure.} % Subcaption 2
    }}
\section{RESULTS}
\resultsTips % Delete when no longer needed
\subsection{Subsection titles should not end in a period}
Per \textit{Cell Genomics}: ``This section should be divided with subheadings. In our view, good subheadings convey information about the findings, so we encourage you to be specific. For example, say "Factor X requires factor Y to function in process Z" rather than "Analysis of factors X and Y using approach Q." We recommend that you use similar language in your figure titles for clarity and structural harmony.''

\subsection{Can Include Optional Initial Paragraph} 
Concise overview of what the reader needs to know about the methods or context (but don’t take too long to get to the real results). \textit{The goal of this work is to demonstrate “X”}. If you use a lot of complicated or easily confused terminology, you can define it here to set the stage. 

\subsection{Declarative result statement}
\textbf{Context} Set up the context and logic for the experiment/analysis. Clearly state the question and hypothesis. Sometimes this will already have been set up by the previous section and can be skipped.

\noindent\textbf{Approach} What was the experiment/analysis, briefly (cite methods section if needed)? Describe key factors of the approach, especially details that apply specifically to this result but not to others using the same method (i.e. specific simulation parameters that vary across results sections).

\noindent\textbf{Result} Clearly state the MAIN result of the experiment/analysis and any relevant statistics. Give a quantification of the effect size, and if you performed a significance test, give the quantification of significance (e.g., p-value or q-value) and test used in parentheses. E.g., \textit{The mean expected 3D divergence is 78\% higher than the observed 3D divergence ($P = 1.8\times10^{-48}$, $t$-test)}. Reference relevant figure panels.

\noindent\textbf{Details} Elaborate on any results that modulate or demonstrate the robustness of your main result. These usually reference supplemental figures that support your main result.

\noindent\textbf{Conclusion} Conservative conclusions drawn from the result. What do these data say about your hypothesis? \textit{This suggests\dots} 

\bigskip

\noindent\textbf{Example} \textit{Given X, we hypothesized that Tony is great. To evaluate/quantify Tony’s greatness, we did X. We found Y (big picture overview of result) (Main text figure reference). We controlled for Z. Tony’s greatness was robust even when considering W (Supplemental figure(s) reference). This suggests that Tony's greatness knows no bounds.}

\subsection{Example results section to demonstrate figure references and placement}
You should insert main text figures and tables into the document as soon as possible after the first mention in the text. We suggest you do this for initial submission even if the journal requests that you put your figures and captions at the end in a separate section. Your reviewers will thank you! Here is an example of how to reference a full figure in-text (\cref{fig:figureSuggestions}) or just a sub-panel (\cref{fig:figureSuggestions:A}). You might also want to reference two figures at the same time like this (\cref{fig:figureSuggestions,fig:tony}). Sometimes you might reference a range of figures like these pair of \textit{Capra} shown in \crefrange{fig:tony}{fig:cutegoat}. You can also reference a range of subpanels like \crefrange{fig:figureSuggestions:A}{fig:figureSuggestions:B}. You can do the same thing for tables (\cref{tab:sampleTable}).

\figureSuggestions

\captionTips 
    \clearpage % can remove if you don't want a new page
    \section{Discussion}
\discussionTips % Delete the tips

\noindent\textbf{First paragraph} Brief summary of the paper that sets up the discussion of the major contributions. You can start by restating the major gap that you addressed (\textit{Despite X, we previously had a limited understanding of Y. Here, we show/demonstrate Z. Together, these findings highlight/provide\dots}) This can parallel the last paragraph of your intro (where you outline the big picture questions and their context for the field) or parallel the results, but this is not required.

\textbf{Middle Paragraph(s)} (1-4 paragraphs) Major contribution/finding/conclusion and its context. What does your result/contribution mean? Who does it speak to and why do they care? What are the implications? Were your hypotheses correct, surprising, contradictory? If the results are not conclusive, why not? How would you perform further work to get to a more conclusive result?

\textbf{Penultimate paragraph(s)} (1-2 paragraphs) Limitations. Identify the broad categories/ themes of limitations and don’t get too particular about the details of each limitation. (i.e., need more samples/ data, model misspecification or over-simplification). Keep it succinct, don’t apologize, but show that you have thought about caveats. Should segue into the future directions. \textit{Although our approach provides many novel benefits, it also has limitations that we hope future work will address.} 

\textbf{Last paragraph} Future directions and final big picture conclusion. This could be broken out into a separate conclusions section. \textit{As X happens, we anticipate that Y will facilitate understanding of Z which allows for W}. Why is this exciting for your audience? Where are the next steps? Note that this is could possibly result in scooped, but even if you are thinking of working on a next step, consider putting it out there, because science for all!! \textit{We encourage future studies to address X}. Be positive about specific future opportunities inspired by your work. It leaves an impression).

    \clearpage % can remove if you don't want a new page
    \section{Methods}
\methodsTips % Delete the tips

\subsection{Some overarching part of your results}
\subsubsection*{Data source 1}
Methods about Data source 1 go here.
\methodsDataTips

\subsubsection*{Analysis 1: descriptive title}
Methods about Analysis 1 go here.
\methodsAnalysisTips

\subsection{Next methods section}
\subsubsection*{Next sub-methods section}
\subsubsection*{Next sub-methods section}


\subsection{Data availability}
The publicly available data used for analysis are available in the following repositories:

The custom datasets we generated are available in the repository “[name here]” available here [\url{link}][citation][accession number].

\subsection{Code availability}
The publicly available code for analysis are available in the following repositories: 

The custom code/software we generated are available in the repository “[name here]” available here [\url{link}][citation][accession number].

\subsection{Acknowledgements}
Acknowledge any help you received from people not in your group. List funding sources. Ask collaborators for information they want to include here.

This work was supported by the National Institutes of Health (NIH) General Medical Sciences award [award number] to [initials].

Acknowledge your computing resources (check their resources): “This work was conducted in part using the resources of the Advanced Computing Center for Research and Education (ACCRE) at Vanderbilt University, Nashville, TN.”

\subsection{Author Contributions}

Summarize the contributions of each author to the project. This section must be approved by all group members. Use the categories, criteria, and style outlined here: \url{http://www.cell.com/pb/assets/raw/shared/guidelines/CRediT-taxonomy.pdf}

\subsection{Competing interests}

The authors declare no competing interests.





    \begin{singlespace}
        \printbibliography % for biblatex
        %\bibliography{library} % for natbib
    \end{singlespace}
    \clearpage

    \section*{Supplementary Information}
    \setcounter{page}{1} % Set page at 1
    \subsection*{Supplemental Text}

Some supplemental text that doesn't fit in the main text or provides extra detail can go here. We will use this space to outline some general tips.

\subsubsection*{Article tips, tricks, and nitpicks}
\generalTips
    \subsection*{Supplemental Figures}
\setcounter{figure}{0} % set figs to start at 1
\renewcommand{\thefigure}{S\arabic{figure}} % start figures with "S"

% See 4maintextFigs.txt for functions and usage to help streamline making figures

%%%% tony %%%%

\generateFig{tony}{tony}{.3} 
    {This is a picture of Tony Capra}
    {Nobody puts Tony in a supplement\dots}
\tony

%%%% cutegoat %%%%
% Example of sideways figure

\generateFig[90]{cutegoat}{cutegoat}{.6} % Make angle 90deg
    {This cute sideways goat wishes you good luck with your manuscript} % bold caption
    {This is how you rotate a figure but not the caption. To rotate the caption with the figure, use the function(s) \texttt{generateSidewaysFig} or \texttt{generateSidewaysFigSubpanels}.} % unbold caption
\cutegoat

\clearpage
    \subsection*{Supplemental Tables}
\setcounter{table}{0} % set tables to start at 1
\renewcommand{\thetable}{S\arabic{table}} % start tables with "S"

%%%% sampleTable %%%%

\generateTab{sampleTable}{supplement/suppl_tabs/sampleTable}{.55}
    {\textbf{\color{sectioncolor}\normalsize Example table.}} % Main caption, note the manual formatting included here. While tedious, this is the easiest way to format supplemental table titles. Just include the code before the title text in any new tables.
    {I use \texttt{Excel2LaTeX.xla} to create easy tables. There are also other online websites. Look for one that is booktabs enabled. Here's good advice about making simple readable tables: \url{https://users.umiacs.umd.edu/~jbg/images/tables.gif}} % Small caption
\sampleTable

% Example of how to make a sideways table
% \generateSidewaysTab{sampleTableSideways}{supplement/suppl_tabs/sampleTable}{.3}
%     {\textbf{\color{sectioncolor}\normalsize Sideways Table.}} % Main caption
%     {Wow! This might be really useful for tables with lots of columns and fewer rows.} % Small caption
% \sampleTableSideways


\clearpage

\end{document}
